The Windows Device Portal Wrapper welcomes contributions from the community.

\section*{Process}


\begin{DoxyEnumerate}
\item \href{https://github.com/Microsoft/WindowsDevicePortalWrapper/issues}{\tt Make a proposal} (either new, or for one of the elements in our backlog)
\item Implement the proposal and its tests.
\item Run Style\+Cop and ensure compliance.
\item Rebase commits to tell a compelling story.
\item Start a pull request \& address comments.
\item Merge.
\end{DoxyEnumerate}

\section*{Proposal}

For things like fixing typos and small bug fixes, you can skip this step.

If your change is more than a simple fix, please don\textquotesingle{}t just create a big pull request. Instead, start by \href{https://github.com/Microsoft/WindowsDevicePortalWrapper/issues}{\tt opening an issue} describing the problem you want to solve and how you plan to approach the problem. This will let us have a brief discussion about the problem and, hopefully, identify some potential pitfalls before too much time is spent.

Note\+: If you wish to work on something that already exists on our backlog, you can use that work item as your proposal.

\section*{Implementation}


\begin{DoxyEnumerate}
\item Fork the repository. Click on the \char`\"{}\+Fork\char`\"{} button on the top right of the page and follow the flow.
\item If your work needs more time, the consider branching off of master else just code in your fork.
\item Ensure your changes check for the appropriate device families (ex\+: Windows Desktop and IoT only).
\item Implement one or more https\+://github.com/\+Microsoft/\+Windows\+Device\+Portal\+Wrapper/blob/master/\+Testing.\+md \char`\"{}tests\char`\"{} to ensure the change works on the target platform(s).
\item Make small and frequent commits that include https\+://github.com/\+Microsoft/\+Windows\+Device\+Portal\+Wrapper/blob/master/\+Testing.\+md \char`\"{}tests\char`\"{} against mock data or manual tests which can be run against real devices.
\item Make sure that all existing https\+://github.com/\+Microsoft/\+Windows\+Device\+Portal\+Wrapper/blob/master/\+Testing.\+md \char`\"{}tests\char`\"{} continue to pass.
\end{DoxyEnumerate}

\section*{Updating code documentation}

The Windows Device Portal Wrapper uses \href{http://www.stack.nl/~dimitri/doxygen/download.html}{\tt Doxygen} to automatically generate code documentation directly from the source code. Any changes to existing or new classes or methods should also update the documentation.


\begin{DoxyEnumerate}
\item Download and install \href{http://www.stack.nl/~dimitri/doxygen/download.html}{\tt Doxygen} (our docs are generated using version 1.\+8.\+11).
\item Open a C\+MD prompt and navigate to your git repository\textquotesingle{}s root directory.
\item Run \textquotesingle{}$<$Doxygen Install Location$>$\textbackslash{}doxygen.\+exe Doc\+Config.\+txt\textquotesingle{}. This will update the files under the html folder relative to the root directory.
\item Include the updated files with your PR.
\end{DoxyEnumerate}

\section*{Run Style\+Cop}

The Windows Device Portal Wrapper uses the \href{http://stylecop.codeplex.com}{\tt Style\+Cop} code analysis tool to ensure code consistency and readability. This step is required for the Windows\+Device\+Portal\+Wrapper folder and is optional (though highly recommended) for test applications.


\begin{DoxyEnumerate}
\item Download and install the latest version of Style\+Cop.
\item Run Style\+Cop analysis on the project (In Visual Studio 2015, select Tools $>$ Run Style\+Cop).
\item Update the source code to address detected issues.
\item Repeat steps 2 and 3 until analysis detects no issues.
\end{DoxyEnumerate}

If there is a Style\+Cop issue that you believe does not need to be enforced, please add the suppression entry either to your code or the Settings.\+Style\+Cop file in the appropriate folder. This will highlight the rule change and allow the community to comment.

\section*{Rebase commits}

The commits in your pull request should tell a story about how the code got from point A to point B. Good stories are edited, so you\textquotesingle{}ll want to rebase your commits so that they tell a good story.

Each commit should build and pass all of the tests. If you want to add new tests for functionality that\textquotesingle{}s not yet written, ensure the tests are added disabled.

Don\textquotesingle{}t forget to run git diff --check to catch those annoying whitespace changes.

Please follow the established \href{https://www.git-scm.com/book/en/v2/Distributed-Git-Contributing-to-a-Project#Commit-Guidelines}{\tt Git convention for commit messages}. The first line is a summary in the imperative, about 50 characters or less, and should not end with a period. An optional, longer description must be preceded by an empty line and should be wrapped at around 72 characters. This helps with various outputs from Git or other tools.

You can update message of local commits you haven\textquotesingle{}t pushed yet using git commit --amend or git rebase --interactivewith reword command.

\section*{Pull request}

Start a Git\+Hub pull request to merge your topic branch into the \href{https://github.com/Microsoft/WindowsDevicePortalWrapper/tree/master}{\tt main repository\textquotesingle{}s master branch}. (If you are a \hyperlink{namespace_microsoft}{Microsoft} employee and are not a member of the \href{https://github.com/Microsoft}{\tt Microsoft organization on Git\+Hub} yet, please contact the D\+DE team via e-\/mail for instructions before starting your pull request. There\textquotesingle{}s some process stuff you\textquotesingle{}ll need to do ahead of time.) If you haven\textquotesingle{}t contributed to a \hyperlink{namespace_microsoft}{Microsoft} project before, you may be asked to sign a \href{https://cla.microsoft.com/}{\tt contribution license agreement}. A comment in the PR will let you know if you do.

The project maintainers will review your changes. We aim to review all changes within three business days. Address any review comments, force push to your topic branch, and post a comment letting us know that there\textquotesingle{}s new stuff to review.

\section*{Merge}

If the pull request review goes well, a project maintainer will merge your changes. Thank you for helping improve the Windows Device Portal Wrapper! 